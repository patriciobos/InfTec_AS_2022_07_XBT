\begin{center}
	\large{\textbf{\textcolor{black}{Dise�o e implementaci�n de etapa de acondicionamiento de se�al para sistema XBT}}}\\
	\vspace{1pc}
	\textbf{Patricio Bos y Mariano Cinquini.}
\end{center}

\bigskip
	\begin{center}
		\textcolor{black}{RESUMEN}	
	\end{center}
	\begin*
		\indent
		\textit{Un batiterm�grafo descartable, XBT por sus siglas en ingl�s (eXpendable BathyThermograph), es
un instrumento utilizado por la Armada Argentina para medir el perfil de temperatura de la
columna de agua en navegaci�n sin afectar las condiciones de operaci�n del buque.  En este Informe T�cnico se incluye una descripci�n funcional del sistema XBT en uso en los buques del Comando de la Flota de Mar (COFM) y de las sondas XBT en particular.  Asimismo, se presentan las caracter�sticas y ensayos sobre distintos dise�os circuitales para la implementaci�n de una etapa de acondicionamiento de se�al que posibilite la adquisici�n y posterior digitalizaci�n de la informaci�n de temperatura que proveen las sondas.  Se documenta la implemetaci�n de uno de los circuitos propuestos con las pruebas de laboratorio para evaluar su desempe�o.}
	\end*

\bigskip
	\begin{center}
		\textcolor{black}{ABSTRACT}
	\end{center}
	\begin*
		\indent
		\textit{An eXpendable BathyThermograph bathythermograph, (XBT), is
an instrument used by the Argentine Navy to measure the temperature profile of the
water column in navigation without affecting the operating conditions of the ship. This Technical Report includes a functional description of the XBT system in use on the ships of the Sea Fleet Command (COFM) and of the XBT probes in particular. Likewise, the characteristics and tests on different circuit designs are presented for the implementation of a signal conditioning stage that enables the acquisition and subsequent digitization of the temperature information provided by the probes. The implementation of one of the proposed circuits is documented with the laboratory tests to evaluate its performance.} 
	\end*

\clearpage